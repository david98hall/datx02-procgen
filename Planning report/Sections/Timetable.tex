
\section{Timetable}
\begin{comment}
\textcolor{red}{
Den här delen av planeringsrapporten beskriver vad som ska göras och när det ska göras. Personer som ska kontaktas bör också stå med här. Datum eller åtminstone veckor då studenterna ska ge delrapporter samt slutgiltiga presentationen ska stå här. Tidsplanen kommer naturligtvis vara rätt grov i början.
\\\\
Det är viktigt att notera att aktiviteterna inom projektet inte kan ske sekventiellt då dessa aktiviteter är beroende av varandra, vilket innebär att ett antal iterationer mellan dem kommer att ske. Endast genom att iterera mellan dem kommer den uppbyggda kunskapen bli utnyttjad på ett bra sätt. Samma tänkande gäller också rapportskrivandet, det vill säga uppdatering av ett avsnitt kräver att även andra uppdateras. Rapportskrivande ska därför ske kontinuerligt under hela projektet.
}
\end{comment}


% Ett sätt att centrera figurer, samt att ge dem labels och captions:
%%\begin{figure}[H]
%\centering
%\includegraphics[scale=0.4]{ENTER_FILEPATH_HERE}
%\caption{This is a caption}
%\label{fig:figure1}
%\end{figure}

\begin{figure}[H]
    \centering
    \includegraphics[width = \textwidth]{Planning report/images/Tidsplan.png}
    \caption{The preliminary timetable of the thesis project. Will be updated as the project progresses.}
    \label{fig:timetable}
\end{figure}