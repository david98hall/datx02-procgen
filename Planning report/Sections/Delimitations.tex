\section{Delimitations}
\textcolor{red}{
Avgränsningarna ska ta upp vilka delar av problemet som inte tas upp i uppsatsen/rapporten, och anledningen till detta. Motivering av avgränsningarna är viktigt.
}
\\\\
\textcolor{blue}{
\textbf{Vad ska vi inte inkludera?}
\begin{itemize}
    %\item Vi gör inget spel, utan ett verktyg för spel. Eftersom vi planerar att utveckla ett verktyg är inte gameplay del av vårt scope.
    \item Staden ska inte vara funktionell, utan helt estetisk. Anledningen till detta är tidsbegränsningar och att spelutvecklare ska kunna lägga till sina egna funktioner till det spel de utvecklar med hjälp av vårt verktyg för stadsgenerering.
    \item Visual fidelity is not a priority; focus will be on developing the algorithm rather than drawing art assets. 
\end{itemize}
}

To better fulfill the purpose of utilizing procedural content generation as a tool for creating custom isometric worlds, this study is solely focused on developing a platform for game development and not on creating actual gameplay. The user will only be able to use the tools and PCG algorithm provided by the platform to generate the layout and contents of the world. It will however not be possible for the user to play in it. 

Generated content is based mainly on realistic aesthetics and not on functionality. Cities, for instance, will only be generated to look realistic, 

will for instance not be generated to behave 


Cities will for instance 


Due to the time constraints and to leave the generated world as