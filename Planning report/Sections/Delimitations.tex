\section{Delimitations}
\label{section:delimitations}

To better utilize procedural content generation as a tool for creating custom worlds in games, this study is solely focused on developing a tool for game development and not on creating actual gameplay. The user will only be able to use it to generate the layout and contents of the world.

The tool will be developed to allow developers to use it for developing most types of games. As of now, it will not be focused on specific game genres, to give the tool a wide range of use cases. This may however change further on, due to the time constraints of the project and possible constraints of the PCG algorithms.

The generated content will mainly be based on credible aesthetics and not on realistic functionality. Cities, for instance, will be generated to have realistic appearance, but will lack behaviors of a real life city such as working freeways, subways and sewage systems. They will also not be realistically accurate of how real cities were built and grew larger. To focus on credibility rather than realism is mainly due to the time constraints of this project. To accurately mimic reality would require a lot of time spent on, for instance, historical research of cities. This was therefore left out of the project. By not generating a too realistic world, we ensure that we don't limit room for creativity for users of the tool.

Visual fidelity is not a priority for the generated content. The focus of the thesis is instead to spend more time developing the algorithms for procedural content generation of major city components rather than drawing sophisticated art assets. This also leaves room for developers to apply their own textures to the generated city and terrain.