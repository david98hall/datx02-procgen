\section{Goal}
The primary goal is to provide a useful and interactive tool for game developers to design cities and surrounding terrains in real-time. Firstly, the tool should be able to generate terrain with elements such as mountains, rivers and lakes that depend on certain probability parameters. The developer should then be able to mark an area where a city should be placed. Through the use of procedural algorithms, a city with, for instance, roads, houses and skyscrapers shall be placed on the terrain. This requires relatively fast algorithms, since it would be frustrating having to wait for the generation to complete.

Another goal is to make the generated cities open for customization, in terms of changing graphical elements, adding functionality or other types of assets. This is important since game developers using the tool should be able to create any city-based game imaginable. The tool should preferably not hinder the creativity of its users.

There are also various other elements that could be added to the tool, for instance simulated humans that are walking or driving cars. Such ideas will, however, have to wait until the earlier mentioned goals of this project have been achieved.

At last, the platform should also provide insights about using PCG in games, to provide a knowledge base for game developers wanting to implement such algorithms in their own designs. This means that different algorithms for generating cities and terrain shall be studied, to discuss their advantages and disadvantages.