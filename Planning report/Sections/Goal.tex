\section{Goals}
\textcolor{red}{
Det här avsnittet är ofta den viktigaste delen av planeringsrapporten (och av den slutgiltiga uppsatsen/rapporten). Den syftar till att identifiera frågan/frågorna som ska tas upp i projektet. Det är viktigt att gruppen gör en problemanalys även om det i projektförslaget redan finns ett problem (en uppgift) specificerat. Anledningen till detta är att det riktiga primära problemet ofta skiljer sig från det i början av uppdragsgivaren/förslagsställaren/kunden föreslagna. Problemanalysen syftar också till att bryta ner problemet/uppgiften i mindre och mer detaljerade delproblem/deluppgifter, vilket också leder till formulering av delsyften. Genom att göra detta får studenterna mycket bättre förståelse för de olika aspekterna av problemet/uppgiften. Utan den här informationen är det omöjligt att identifiera vilken information som behövs, vilka informationskällor som behövs och lämpliga tillvägagångssätt.
\\\\
En bra problemanalys som identifierar delproblem/deluppgifter och delsyften vilar i många fall på användning av teorier och modeller från litteraturen. En litteraturgenomgång bör därför genomföras tidigt i processen.
}

\textcolor{blue}{
\textbf{Projektmål (syfte, egenskaper, funktionalitet, etc)}
\begin{itemize}
    \item “Fungerande” stad
    \item Logiskt (e.g. vägnät)
    \item Använder oss utav PCG
    \item Algoritmer för olika typer av städer
    \item Interaktion med staden
\end{itemize}
\textbf{Effektmål (nytta, vad det ska leda till)}
\begin{itemize}
    \item Grund till spel för indie-utvecklare
    \item Insikter om att använda PCG i spel
\end{itemize}
}

In order for the tool to be useful for game developers, the goal is to provide an interactive way to design cities and surrounding terrain in real time. The tool shall be able to first generate a terrain with mountains, rivers and lakes depending on certain parameters (\textit{what parameters?}). The designer shall then be able to mark an area where a city is to be placed and through the use of procedural algorithms, a city with roads, houses and skyscrapers is placed on the terrain. 

The goal is also to provide insights about using PCG in games to provide a knowledge base for developers wishing to implement algorithms of their own design in games they create. This means we shall study different algorithms of generating cities and terrain to discuss their advantages and disadvantages.



