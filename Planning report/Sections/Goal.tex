\section{Goals}
\begin{comment}

\textcolor{red}{
Det här avsnittet är ofta den viktigaste delen av planeringsrapporten (och av den slutgiltiga uppsatsen/rapporten). Den syftar till att identifiera frågan/frågorna som ska tas upp i projektet. Det är viktigt att gruppen gör en problemanalys även om det i projektförslaget redan finns ett problem (en uppgift) specificerat. Anledningen till detta är att det riktiga primära problemet ofta skiljer sig från det i början av uppdragsgivaren/förslagsställaren/kunden föreslagna. Problemanalysen syftar också till att bryta ner problemet/uppgiften i mindre och mer detaljerade delproblem/deluppgifter, vilket också leder till formulering av delsyften. Genom att göra detta får studenterna mycket bättre förståelse för de olika aspekterna av problemet/uppgiften. Utan den här informationen är det omöjligt att identifiera vilken information som behövs, vilka informationskällor som behövs och lämpliga tillvägagångssätt.

En bra problemanalys som identifierar delproblem/deluppgifter och delsyften vilar i många fall på användning av teorier och modeller från litteraturen. En litteraturgenomgång bör därför genomföras tidigt i processen.
}

\textcolor{blue}{
\textbf{Projektmål (syfte, egenskaper, funktionalitet, etc)}
\begin{itemize}
    \item “Fungerande” stad
    \item Logiskt (e.g. vägnät)
    \item Använder oss utav PCG
    \item Algoritmer för olika typer av städer
    \item Interaktion med staden % Kanske inte relevant längre
\end{itemize}
\textbf{Effektmål (nytta, vad det ska leda till)}
\begin{itemize}
    \item Grund till spel för indie-utvecklare
    \item Insikter om att använda PCG i spel
\end{itemize}
}
\end{comment}

The primary goal is to provide a useful and interactive tool for game developers to design cities and surrounding terrains in real-time. Firstly, the tool should be able to generate terrain with elements such as mountains, rivers and lakes that depend on certain probability parameters. The developer should then be able to mark an area where a city should be placed. Through the use of procedural algorithms, a city with, for example, roads, houses and skyscrapers shall be placed on the terrain. This requires fast algorithms, since it would be frustrating having to wait too long for the generation to complete.

Another goal is to make the generated cities open for customization in terms of changing graphical elements and adding functionality. This is important since game developers using the tool should be able to create any city-based game imaginable. The tool should preferably not hinder the creativity of its users at all.

At last, the platform should also provide insights about using PCG in games to provide a knowledge base for game developers wanting to implement such algorithms in their design. This means that different algorithms for generating cities and terrain shall be studied, to discuss their advantages and disadvantages.

There are also a lot of other elements that could be added to the tool, such as adding simulated humans that are, for example, walking or driving cars. Such ideas will, however, have to wait until the earlier mentioned goals of this project have been achieved.