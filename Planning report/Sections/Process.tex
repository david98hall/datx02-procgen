\section{Process}
\textcolor{red}{
Hur gruppen har tänkt sig att genomföra arbetet är val av metod. I konstruktionsinriktade projekt kan detta tyckas vara självklart, men det kan även i detta fall finnas viktiga metodval. Helt litteraturbaserade kandidatarbeten är också genomförbara men även en litteraturstudie ska ha en ordnad och strukturerad arbetsprocess och metodik.
\\\\
Metodavsnittet bör också beskriva hur data ska samlas in och hur det konstateras hur väl projektets mål har uppfyllts. I praktiska projekt kan detta vara genom mätningar av olika typer. Det kan också vara genom datorsimuleringar. Vilka aspekter är viktiga för att veta om syftet med projektet har uppnåtts? Datainsamling kan också vara en del av en testning eller annan utvärdering av den produkt som tas fram i ett konstruktionsinriktat projekt.
\begin{itemize}
    \item Antal studieobjekt/testfall och hur de väljs?
    \item Typ av undersökningsmetod/testmetod? 
    \item Hur insamlade data/testresultat ska analyseras och presenteras? 
    \item Hur ser processen ut för litteraturarbetet?
\end{itemize}
}
\textcolor{blue}{
\begin{itemize}
    \item Projektmodell: Scrum
    \item Utvecklingsverktyg: Unity
    \item Programmeringsspråk: C\#
\end{itemize}
}

As the project requires a good understanding of the concept of PCG and its varied implementations, an initial objective is to find and study material pertaining to the subject. This includes commonly used algorithms and how they were implemented in games and other generators. These would then be evaluated for the purpose of generating isometric city environments.

An early candidate for generating road networks was L-systems, due to their branching nature. L-systems are typically used to generate vegetation and were introduced to model natural structures, but have also seen use in city generation algorithms.~\cite{yoav-pascal} Another possible method to explore is Voronoi diagrams, which has many application areas but could be used for determining the position of the roads.

The main tools to use for the project were decided on early; the generator would be built using Unity since the main focus of the project are the PCG algorithms and not the development of a game engine. As a result of this, the code will be written in C\# due to its close connection to Unity and the fact that it's a language all individuals in the group had experience with. For structuring the objectives and sub goals of the implementation, the Scrum model was chosen due to previous good experiences with it.
