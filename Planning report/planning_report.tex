\documentclass[a4paper,11pt]{article}

% ---- Paket ----
\usepackage[format=plain,font=normal]{caption}
\usepackage{lmodern}
\usepackage[T1]{fontenc}
\usepackage[letterspace=150]{microtype}
\usepackage{color}
\usepackage{lipsum,mathtools}
\usepackage{fancyhdr}
\usepackage[utf8]{inputenc}
\usepackage{geometry}
\usepackage{graphicx}
\usepackage{changepage}
\usepackage{placeins}
\usepackage{parskip}
\usepackage[english]{babel}
\usepackage{float}
\usepackage{booktabs}
\usepackage{enumitem}
\usepackage{amsmath}
\usepackage{blindtext}
\usepackage{scrextend}
\usepackage[normalem]{ulem}
\usepackage{float}
\useunder{\uline}{\ul}{}
\usepackage{titlesec}
\usepackage{amsmath}
\usepackage{amssymb}
\usepackage[export]{adjustbox}
\usepackage{biblatex}
\usepackage{url}
\usepackage{csquotes}

% ---- Inställningar ----
\setlength{\parskip}{1em}   % whitespace between paragraphs
\setlength{\parindent}{0em} % no indentation in new paragraphs
\numberwithin{equation}{section}
\newcommand\tab[1][0.5cm]{\hspace*{#1}}
\addbibresource{references.bib}

% ---- Färger ---
\definecolor{light-gray}{gray}{0.40}
\definecolor{mygreen}{RGB}{28,172,0}
\definecolor{mylilas}{RGB}{170,55,241}
\lhead
\rhead

\begin{document}
\pagestyle{empty}

\begin{titlepage}
\title{ \Huge{Procedurally Generated Isometric Tool for Cities in Game Development} \noindent\makebox[\linewidth]{\textcolor{light-gray}{\rule{\paperwidth}{0.1pt}}}}
\author{\LARGE{{Grupp 85}} \\\\ 
[0.30cm] Alma Eriksson, David Hall\\
[0.30cm] David Hultsten, William Johnsson\\
[0.30cm] Ludvig Liljeqvist, Ludwig Lundholm Hultqvist
\\[0.1cm]}
\maketitle
\thispagestyle{empty}
\end{titlepage}

\clearpage
\maketitle
\tableofcontents
\newpage
\pagestyle{fancyplain}
\section{Background}
\begin{comment}
\textcolor{red}{
Bakgrund ska innehålla en motivering till varför det valda ämnet är intressant ur akademisk synvinkel och/eller ur tekniskt perspektiv eller i förekommande fall ur kundens/uppdragsgivarens perspektiv. I vissa fall ska den här rubriken inkludera en kort historik över ämnet. Efter att ha läst bakgrunden ska alla läsare förstå varför ämnet är relevant. Följande frågeställningar bör vara aktuella:
\begin{itemize}
    \item Vad är ämnet/problemet som ska undersökas?
    \item Varför har ämnet/problemet uppkommit?
    \item Varför och för vem är det ett relevant eller intressant ämne/problem?
    \item Kan det specifika ämnet/problemet relateras till en mer generell diskussion?
\end{itemize}
}
\end{comment}

Improvements in computational power has resulted in graphical applications containing more detail and realism. One way to provide this is through the use of Procedural Content Generation (PCG) which can be defined as ''the algorithmic creation of game content with limited or indirect user input'' \cite[p. 1]{shaker2016procedural}. This means that PCG can be used to either create content fully automatically or with a low degree of human assistance.

\subsection{Why Procedural Content Generation is useful}

There are several arguments for using PCG, four of which are \cite[pp. 141-142]{search-based_pcg}:
\begin{enumerate}
    \item PCG is memory efficient since content only has to be generated when it is needed.
    \item Using PCG is relatively effortless in terms of development time and expenses.
    \item PCG has a potential for being used to generate whole games of new kinds and with unlimited replayability.
    \item PCG can be used as a tool to make us humans think ''out of the box'' since it can provide us with content we probably would not have come up with on our own.
\end{enumerate}

 
Another reason for game developers to use PCG methods is that the algorithms are capable of creating vast amounts of game assets at a faster rate than 3D-artists. This will result in a potential cost reduction for the production as entire teams won't be needed to work on this particular aspect of the game. On top of this, the algorithms can be designed in a way that the world is generated in real-time, creating an endless world for the player to explore. However, special care must be taken when writing PCG algorithms to create a visually interesting and diverse world that does not break gameplay.



%%%%%%%%%%%%%%%%%%%%%%%%%%%%%%%%%%%%%%%%%%%%%%%%%%%%%%%%%%%%%%%%%%%%%%%%%%%%%%%%%%%%%%%
PCG may also be used as a complementary tool for worlds built manually. When utilized in this fashion, PCG does not generate new assets, but rather hides and reveals existing ones on demand. This is a necessary tool in games with larger worlds. In such worlds, objects need to be viewed at differing levels of detail, depending on distance to the player. When far away, many objects fit on the player's screen. Rendering them all in full detail can be detrimental to performance, and also unnecessary because of restraints on screen resolution. With the help of PCG, smaller assets can be hidden when viewed from a great distance, and the detail of larger ones can be downscaled. \cite[p. 57]{shaker2016procedural}

%%%%%%%%%%%%%%%%%%%%%%%%%%%%%%%%%%%%%%%%%%%%%%%%%%%%%%%%%%%%%%%%%%%%%%%%%%%%%%%%%%%%%%%
It is important to note that PCG is also used in industries other than game development. Two examples of this are SpeedTree~\cite{SpeedTree}, a tool for generating virtual foilage that is used by architects and animators and Esri CityEngine~\cite{CityEngine}, modeling software for creating immersive urban environments based on geographical data.

%%%%%%%%%%%%%%%%%%%%%%%%%%%%%%%%%%%%%%%%%%%%%%%%%%%%%%%%%%%%%%%%%%%%%%%%%%%%%%%%%%%%%%%
%---------------- examples of games ----

\subsection{Examples of games utilizing PCG}

A few examples of games using PCG are \textit{Diablo} for generating maps, \textit{Spore} for creature animations and \textit{Minecraft} for landscapes and caves \cite[p. 5]{shaker2016procedural}. Perhaps the greatest example of a game utilizing PCG is \textit{No Man's Sky}, which procedurally generates everything from vegetation and creatures to landscapes and planets and thus creates a virtually endless and unique universe for its players to explore \cite{MIT_No_Mans_Sky}.

As of May 2019 and after 10 years of its official release, \textit{Minecraft} has sold over 176 million copies \cite{Minecraft_10ys}. It is a valid theory that the random generation of worlds using PCG has something to do with the game's overwhelming success.

% ------------------ cities----------------

\subsection{Using PCG to create cities} %trovärdigt
When creating a city using PCG in games, 1it is fundamental to make it functional and realistic city for the developers to use in their games. A realistic city could suit more developers and productions due to its simplicity and recognition and can easily be changed into its final desired city. This can be done by studying urbanization of a larger city and its developments when its populations are increased, this is called Large-scale Urban Development(LUD) \cite{UrbanCity}. LUDs are a way to make sure that a growing population has housing and its housing is developed with the environment in mind. Buildings built today are now taller than before, so each new building can house more people on the same square meters of the ground as before \cite{UrbanCity}. A city is not only housing for the population, but a city is also made whole by using architecture to form the surroundings, such as culture, parks, roads, public transportation, policies, hospitals and styles of its buildings \cite{CityArchitecture}. 

% skriva mer om grids och cul de sac 

% mer om spelen, hur pcg används
\section{Purpose}
\begin{comment}
\textcolor{red}{
Syftet specificerar vad projektet är tänkt att resultera i och vilken typ av resultat som kommer att uppnås. Ett projekt kan ha flera syften som är relaterade till de ämnen/problem som presenteras i bakgrunden. I de flesta fall är det dock lämpligt att ha endast ett generellt syfte, som sedan bryts ner i mer detaljerade delar längre fram i kandidatarbetets process och uppsats/rapport.
}
\end{comment}

This thesis focuses on algorithms for Procedural Content Generation; more specifically, a tool for automating the creation of realistic cities to make game development cost-effective and less time-consuming.
\section{Goals}
\textcolor{red}{
Det här avsnittet är ofta den viktigaste delen av planeringsrapporten (och av den slutgiltiga uppsatsen/rapporten). Den syftar till att identifiera frågan/frågorna som ska tas upp i projektet. Det är viktigt att gruppen gör en problemanalys även om det i projektförslaget redan finns ett problem (en uppgift) specificerat. Anledningen till detta är att det riktiga primära problemet ofta skiljer sig från det i början av uppdragsgivaren/förslagsställaren/kunden föreslagna. Problemanalysen syftar också till att bryta ner problemet/uppgiften i mindre och mer detaljerade delproblem/deluppgifter, vilket också leder till formulering av delsyften. Genom att göra detta får studenterna mycket bättre förståelse för de olika aspekterna av problemet/uppgiften. Utan den här informationen är det omöjligt att identifiera vilken information som behövs, vilka informationskällor som behövs och lämpliga tillvägagångssätt.
\\\\
En bra problemanalys som identifierar delproblem/deluppgifter och delsyften vilar i många fall på användning av teorier och modeller från litteraturen. En litteraturgenomgång bör därför genomföras tidigt i processen.
}

\textcolor{blue}{
\textbf{Projektmål (syfte, egenskaper, funktionalitet, etc)}
\begin{itemize}
    \item “Fungerande” stad
    \item Logiskt (e.g. vägnät)
    \item Använder oss utav PCG
    \item Algoritmer för olika typer av städer
    \item Interaktion med staden
\end{itemize}
\textbf{Effektmål (nytta, vad det ska leda till)}
\begin{itemize}
    \item Grund till spel för indie-utvecklare
    \item Insikter om att använda PCG i spel
\end{itemize}
}

In order for the tool to be useful for game developers, the goal is to provide an interactive way to design cities and surrounding terrain in real time. The tool shall be able to first generate a terrain with mountains, rivers and lakes depending on certain parameters (\textit{what parameters?}). The designer shall then be able to mark an area where a city is to be placed and through the use of procedural algorithms, a city with roads, houses and skyscrapers is placed on the terrain. 

The goal is also to provide insights about using PCG in games to provide a knowledge base for developers wishing to implement algorithms of their own design in games they create. This means we shall study different algorithms of generating cities and terrain to discuss their advantages and disadvantages.




\section{Delimitations}
\label{section:delimitations}

To better utilize procedural content generation as a tool for creating custom worlds in games, this study is solely focused on developing a tool for game development and not on creating actual gameplay. The user will only be able to use it to generate the layout and contents of the world.

The tool will be developed to allow developers to use it for developing most types of games. As of now, it will not be focused on specific game genres, to give the tool a wide range of use cases. This may however change further on, due to the time constraints of the project and possible constraints of the PCG algorithms.

The generated content will mainly be based on credible aesthetics and not on realistic functionality. Cities, for instance, will be generated to have realistic appearance, but will lack behaviors of a real life city such as working freeways, subways and sewage systems. They will also not be realistically accurate of how real cities were built and grew larger. To focus on credibility rather than realism is mainly due to the time constraints of this project. To accurately mimic reality would require a lot of time spent on, for instance, historical research of cities. This was therefore left out of the project. By not generating a too realistic world, we ensure that we don't limit room for creativity for users of the tool.

Visual fidelity is not a priority for the generated content. The focus of the thesis is instead to spend more time developing the algorithms for procedural content generation of major city components rather than drawing sophisticated art assets. This also leaves room for developers to apply their own textures to the generated city and terrain.
\section{Process}
\textcolor{red}{
Hur gruppen har tänkt sig att genomföra arbetet är val av metod. I konstruktionsinriktade projekt kan detta tyckas vara självklart, men det kan även i detta fall finnas viktiga metodval. Helt litteraturbaserade kandidatarbeten är också genomförbara men även en litteraturstudie ska ha en ordnad och strukturerad arbetsprocess och metodik.
\\\\
Metodavsnittet bör också beskriva hur data ska samlas in och hur det konstateras hur väl projektets mål har uppfyllts. I praktiska projekt kan detta vara genom mätningar av olika typer. Det kan också vara genom datorsimuleringar. Vilka aspekter är viktiga för att veta om syftet med projektet har uppnåtts? Datainsamling kan också vara en del av en testning eller annan utvärdering av den produkt som tas fram i ett konstruktionsinriktat projekt.
\begin{itemize}
    \item Antal studieobjekt/testfall och hur de väljs?
    \item Typ av undersökningsmetod/testmetod? 
    \item Hur insamlade data/testresultat ska analyseras och presenteras? 
    \item Hur ser processen ut för litteraturarbetet?
\end{itemize}
}
\textcolor{blue}{
\begin{itemize}
    \item Projektmodell: Scrum
    \item Utvecklingsverktyg: Unity
    \item Programmeringsspråk: C\#
\end{itemize}
}

As the project requires a good understanding of the concept of PCG and its varied implementations, an initial objective is to find and study material pertaining to the subject. This includes commonly used algorithms and how they were implemented in games and other generators. These would then be evaluated for the purpose of generating isometric city environments.

An early candidate for generating road networks was L-systems, due to their branching nature. L-systems are typically used to generate vegetation and were introduced to model natural structures, but have also seen use in city generation algorithms.~\cite{yoav-pascal} Another possible method to explore is Voronoi diagrams, which has many application areas but could be used for determining the position of the roads.

The main tools to use for the project were decided on early; the generator would be built using Unity since the main focus of the project are the PCG algorithms and not the development of a game engine. As a result of this, the code will be written in C\# due to its close connection to Unity and the fact that it's a language all individuals in the group had experience with. For structuring the objectives and sub goals of the implementation, the Scrum model was chosen due to previous good experiences with it.

\section{Ethics}
\begin{comment}
\textcolor{red}{
I planeringsrapporten förväntas gruppen skriva en kortare text där gruppen bedömer om samhälleliga och etiska aspekter behöver beaktas och analyseras vidare i uppsatsen/rapporten. Gruppen använder sig med fördel av bilaga 7 som stöd samt de digitala resurser som finns på Studentportalens sidor om kandidatarbetet.
}
\end{comment}

The tool that is going to be developed will probably not affect society in a particularly negative way. The only thing that may be worth mentioning is that some people might lose their jobs if companies start using this tool. People working with designing cities might be affected since the tool will automate their jobs. On the other hand, our tool will make game development easier for smaller development teams as well as for hobbyists. Besides, automated generation has been widespread in graphics for decades, and this tool is merely meant to serve as a basis upon which games can be developed.

%skriv mer om det fungerar mer som en stad, som tåg, bussar och inte bara vägtrafik.

\section{Timetable}
\textcolor{red}{
Den här delen av planeringsrapporten beskriver vad som ska göras och när det ska göras. Personer som ska kontaktas bör också stå med här. Datum eller åtminstone veckor då studenterna ska ge delrapporter samt slutgiltiga presentationen ska stå här. Tidsplanen kommer naturligtvis vara rätt grov i början.
\\\\
Det är viktigt att notera att aktiviteterna inom projektet inte kan ske sekventiellt då dessa aktiviteter är beroende av varandra, vilket innebär att ett antal iterationer mellan dem kommer att ske. Endast genom att iterera mellan dem kommer den uppbyggda kunskapen bli utnyttjad på ett bra sätt. Samma tänkande gäller också rapportskrivandet, det vill säga uppdatering av ett avsnitt kräver att även andra uppdateras. Rapportskrivande ska därför ske kontinuerligt under hela projektet.
}
%\section{Budget}
\textcolor{red}{
Ifall vi planerar på att spendera pengar.
}
\newpage
\section{References}
\printbibliography[heading=none]
\end{document}

% ---- Föreläsning om planeringsrapport ---
% ska finnas på canvas
% specifikt: mål/syfte/metod/plan
% håll fast vid givet upplägg
% problemavsnitt och metod ska vara välskrivna
% titta inte på tidigare planeringsrapporter
% skriv revisionsversion för varje "större" ändring. Visa på varje sida (tror jag att ha sade)
% strikt formel slutrapport

% Bakgrund:
% Viktigt - varför/hur/för vem relevant + vad ska åstakommas

% Syfte:
% Vilka konkreta resultat som ska förmedladas. Relativt kort. 

% Problem / Uppgift:
% Vilka frågor ska besvaras

% Avgränsningar:
% Antingen efter syfte eller i efter slutsats, optimalt med båda.

% Metod / genomförande:
% Testa tankar, iterativt, uppdatera efter hand

% Samhälliga och etiska aspekter
% Ifall det inte finns tydliga etiska aspekter. Motivera varför det inte finns

% Tidsplan
% Konkret. I detalj närmaste tiden och i grova drag längre fram. 
% Milstolpar. 
% Sammanfatta i ett Gantt-schema. 